\documentclass[12pt,a4paper,twoside]{article}

%------------------------------ encoding
\usepackage[utf8]{inputenc}

%------------------------------ page layout
\usepackage[left=3cm,right=3cm,top=4cm,bottom=5cm]{geometry}

%------------------------------ clickable toc and links styles
\usepackage{hyperref}
\hypersetup{
    colorlinks,
    citecolor=black,
    filecolor=black,
    linkcolor=black,
    urlcolor=black
}

\begin{document}

\begin{titlepage}
    \centering
    \vspace*{\fill}

    \vspace*{0.5cm}
    
    \huge ACULEI
    
    \vspace*{1cm}
    \small Brajucha Filippo, Dinelli Michele, Hanna Yossef
    
    \vspace*{0.5cm}
    \small Dicembre 2023
    
    \vspace*{\fill}
\end{titlepage}

%\maketitle

\newpage

\section*{Abstract}

Questo report tratta della realizzazione di un'intelligenza artificiale al servizio di un archivio fotografico 
online. Il progetto si chiama ACULEI e nasce dall'idea di Tobia Faverio, un fotografo nato a Milano e cresciuto 
in Umbria. ACULEI è una raccolta molta vasta di fotografie scattate da foto-trappole sparse per un boschetto nel 
cuore dell'Umbria. Le foto-trappole sono state disposte da Tobia durante i primi mesi della Pandemia da virus 
SARS-CoV-2 per monitorare quale specie animali frequentassero quelle zone. 

Sono stati prodotti numerosi scatti (ad oggi più di 17 000) che ritraggono vari animali come istrici, cinghiali, 
caprioli e addirittura lupi. ACULEI dunque vuole offrire un'esperienza virtuale e interattiva per esplorare 
l'archivio fotografico nato dalla raccolta di scatti. 

Per rendere migliore l'esperienza è stata sviluppata un'intelligenza artificiale che possa guidare il visitatore 
dell'archivio, offrendogli un percorso selezionato così da scoprire le fotografie simili o correlate.
 
\newpage

\tableofcontents

\newpage
\section{Introduzione}

\subsection*{Descrizione del Problema}
Navigare tra le fotografie che abbiamo a disposizione seguendo un senso logico, spaziando tra immagini 
correlate per periodo o fase lunare o animale o altri tipi di relazione.\\
Sostanzialmente individuare un percorso più o meno strutturato e complesso che permetta di vedere più 
fotografie possibili.
\subsubsection*{Motivazione del problema e rilevanza}
Questa commissione è stata effettuata dal propiretario delle foto che possiede anche le camere che le hanno
scattate, per realizzare una pubblicazione artistica in modo da rendere fruibili da tutti un patrimonio 
enorme rappresentato da queste foto che altrimenti resterebbe solo privato e sarebbe quasi uno spreco.\\
L'idea è nata durante la pandemia, quando queste foto-trappole sono state piazzate e si è sviluppata fino ad 
arrivare a questa idea finale.
\subsubsection*{Pubblico interessato del problema}
Il target individuato potrebbe essere sicuramente collegato ad esperti del settore naturalistico che possono 
utilizzare le foto messe a disposizione per studi ma anche ad apassionati di natura che possono esplorare 
la fauna selvatica semplicemente con un click.\\
Il pubblico interessato varia molto perchè il lavoro è tanto e i possibili utilizzi pure; oltre all'utente 
finale, infatti, potrebbero esserci altri utenti interessati, come coloro che possono sfruttare il data-set 
che abbiamo realizzato, opure coloro che possono utilizzare e migliorare la nostra clusterizzazione.
\subsubsection*{Benefici di una soluzione}
Una soluzione curata e precisa potrebbe essere sicuramente utile e interessante per l'intrattenimento 
dell'utente finale, noi, però, l'abbiamo pensata anche in ottica di ricerca, infatti può essere utile per 
osseravre e tracciare dei comportamentei particolari degli animali selvatici.

\subsection*{Soluzione Proposta}
\subsubsection*{Approccio alla soluzione}
\subsubsection*{Sfide informatiche affrontate}
\subsubsection*{Rassegna della letteratura}
\subsubsection*{Divisione dei compiti nel gruppo}
\subsubsection*{Risultati ottenuti in sintesi}


\newpage
\section{Metodo Proposto}

\subsection*{Scelta della Soluzione}
\subsubsection*{Alternativi considerati e giustificazioni della scelta}
\subsubsection*{Metodologia per la misurazione delle performance}


\newpage
\section{Risultati Sperimentali}

\subsection*{Dimostrazione e Tecnologie}
\subsubsection*{Istruzione per la dimostrazione}
\subsubsection*{Tecnologie e versioni usate (riproducibilità)}

\subsection*{Risultati}
\subsubsection*{Risultati della configurazione migliore}
\subsubsection*{Studio di ablazione: confronto tra configurazioni}
\subsubsection*{Studio di comparazione con letteratura}


\newpage
\section{Discussione e Conclusioni}

\subsection*{Discussione dei Risultati}
\subsubsection*{Analisi delle performance rispetto alle aspettative}

\subsection*{Validità del Metodo}
\subsubsection*{Valutazione se il metodo rispetta le aspettative}

\subsection*{Limitazione e Maturità}
\subsubsection*{Limiti di applicabilità e bias}

\subsection*{Lavori Futuri}
\subsubsection*{Proposte per avanzare il progetto}

\end{document}