\documentclass[12pt,a4paper,twoside]{article}

%------------------------------ encoding
\usepackage[utf8]{inputenc}

%------------------------------ page layout
\usepackage[left=3cm,right=3cm,top=4cm,bottom=5cm]{geometry}

%------------------------------ clickable toc and links styles
\usepackage{hyperref}
\hypersetup{
    colorlinks,
    citecolor=black,
    filecolor=black,
    linkcolor=black,
    urlcolor=black
}

%------------------------------ font macro
\newcommand*{\courierfont}{\fontfamily{pcr}\selectfont}

%------------------------------ references
\usepackage{biblatex}
\addbibresource{bib.bib}

%------------------------------ table
\usepackage{tabularx}

\begin{document}

\begin{titlepage}
    \centering
    \vspace*{\fill}

    \vspace*{0.5cm}
    
    \huge ACULEI
    
    \vspace*{2.5cm}
    
    \vspace*{1cm}

    \begin{minipage}{\textwidth}
        \centering
        \begin{tabular}{ccc}
            \small \textbf{Brajucha Filippo} & \small \textbf{Dinelli Michele} & \\
            \small University of Bologna & \small University of Bologna & \\
            \scriptsize \courierfont{\href{mailto:filippo.brajucha@studio.unibo.it}{filippo.brajucha@studio.unibo.it}} & \scriptsize \courierfont{\href{mailto:michele.dinelli5@studio.unibo.it}{michele.dinelli5@studio.unibo.it}} & \\
        \end{tabular}
    \end{minipage}

    \vspace*{1cm}

    \begin{minipage}{\textwidth}
        \centering
        \begin{tabular}{ccc}
            \small \textbf{Hanna Youssef} \\
            \small University of Bologna \\
            \scriptsize \courierfont{\href{mailto:youssefawni.hanna@studio.unibo.it}{youssefawni.hanna@studio.unibo.it}} \\
        \end{tabular}
    \end{minipage}
    
    \vspace*{\fill}
\end{titlepage}

%\maketitle

\tableofcontents

\newpage

\section*{Abstract}
Questo report descrive il processo di realizzazione di un dataset che raccoglie i dati di fotografie di animali scattate automaticamente da foto-trappole situate nei boschi dell'Umbria. Realizzare il dataset ha lo scopo di raccogliere una vasta quantità di dati, così da metterli al servizio di tecniche di intelligenza artificiale, al fine di raggruppare e potenzialmente classificare le fotografie. Si esplorano due approcci: il primo prevede l'apprendimento non supervisionato basato su algoritmi di clusterizzazione, il secondo invece si basa sull'utilizzo di modelli pre-addestrati (eventualmente valutando operazioni di fine-tuning) al fine di classificare le fotografie in base all'animale ritratto. Negli ultimi anni la disciplina nota come computer vision ha fatto passi enormi, grazie a questa branchia dell'informatica è  è possibile manipolare dati complessi come le fotografie estraendo da esse conoscenza. \\ Le informazioni ottenute dalle fotografie, le relazioni e i pattern nascosti che vengono identificati dall'intelligenza artificiale sono utilizzati per generare un'esperienza interattiva all'interno di un archivio fotografico chiamato \textbf{aculei}. \\ Il progetto aculei nasce dall'idea di un fotografo Milanese che ha distribuito varie foto-trappole automatiche nei boschi che circondano la casa dove è cresciuto in Umbria. La volontà di pubblicare le fotografie e rendere l'esperienza sull'archivio ai-driven sono le motivazioni che lo spingono a ideare il progetto aculei.

\section{Introduzione}

\subsection{Descrizione del Problema}
Abbiamo a nostra disposizione una raccolta di fotografie (circa 40GB) scattate da alcune foto-trappole 
sparse per un appezzamento di terreno abbastanza vasto in Umbria, queste foto ci sono state passate dal 
proprietario del terreno che le ha raccolte negli ultimi 4 anni e ora vuole creare un sistema per 
renderle fruibili a tutti.\\
L'idea iniziale era quella di creare un AI che fosse in grado di riconoscere l'animale immortalato e 
riuscisse a "consigliare" delle altre immagini che ritraessero lo stesso animale. Analizzando il problema 
abbiamo deciso di provare a creare un "viaggio" tra le immagini ma senza riconoscimento che risulterebbe 
troppo complesso e ingegnoso, piuttosto creare un percorso attraverso la clusterizzazione.\\ 
Lo scopo finale è quindi quello di navigare tra le fotografie che abbiamo a disposizione seguendo un 
senso logico, spaziando tra immagini correlate per periodo o fase lunare o animale o altri tipi di 
relazione.\\
Sostanzialmente individuare un percorso più o meno strutturato e complesso che permetta di vedere più 
fotografie possibili.
\subsubsection{Motivazione del problema e rilevanza}
Questa commissione è stata effettuata dal propiretario delle foto che possiede anche le camere che le hanno
scattate, per realizzare una pubblicazione artistica in modo da rendere fruibili da tutti un patrimonio 
enorme rappresentato da queste foto che altrimenti resterebbe solo privato e sarebbe quasi uno spreco.\\
L'idea è nata durante la pandemia, quando queste foto-trappole sono state piazzate e si è sviluppata fino ad 
arrivare a questa idea finale.
\subsubsection{Pubblico interessato del problema}
Il target individuato potrebbe essere sicuramente collegato ad esperti del settore naturalistico che possono 
utilizzare le foto messe a disposizione per studi ma anche ad apassionati di natura che possono esplorare 
la fauna selvatica semplicemente con un click.\\
Il pubblico interessato varia molto perchè il lavoro è tanto e i possibili utilizzi pure; oltre all'utente 
finale, infatti, potrebbero esserci altri utenti interessati, come coloro che possono sfruttare il data-set 
che abbiamo realizzato, opure coloro che possono utilizzare e migliorare la nostra clusterizzazione.
\subsubsection{Benefici di una soluzione}
Una soluzione curata e precisa potrebbe essere sicuramente utile e interessante per l'intrattenimento 
dell'utente finale, noi, però, l'abbiamo pensata anche in ottica di ricerca, infatti può essere utile per 
osseravre e tracciare dei comportamentei particolari degli animali selvatici.

\subsection{Soluzione Proposta}
La soluzione proposta è interamente incentrata nella creazione di un percorso interattivo generato tramite 
la clusterizzazione delle immagini utilizzando diverse features.
\subsubsection{Approccio alla soluzione}
Per arrivare alla soluzione abbiamo eseguito diversi tenativi con molteplici tecnologie, più o meno 
sofisticate, che ci hanno restituito sia ottimi risultati che pessimi.\\
Per ottenere la soluzione ottima a cui siamo arrivati abbiamo attraversato un percorso con molti ostacoli, 
è stato fondamentale mantere la comunicazione nel gruppo. Molto spesso ci siamo trovati a discutere su 
strade alternative o possibili soluzione a problemi, sicuramente senza la discussione tra di noi non 
saremmo mai arrivati ad una conclusione rapidamente e avremmo impiegato molto più tempo ad affrontare le 
insidie.\\
La prima soluzione a cui avevamo pensato è stata quella di addestrare un AI con diverse immagini per poi 
cercare di ottenere un output soddisfacente ed efficiente in cui essa avrebbe dovuto riconoscere lo stesso 
animale presente nella foto. Il processo è stato piuttosto complesso e vedendo gli scarsi risultati abbiamo 
abbandonato quasi subito questa idea.\\
Dopo aver ottenuto dei risultati poco soddisfacenti ci siamo riuniti con il Professore per cercare di 
trovare insieme una soluzione più congeniale, lui ci ha aiutato molto nell'esplorare altre idee pensando un 
po fuori dagli schemi, osservando il problema da fuori e ri-organizzando il materiale a disposizione in 
modo da lavorare con maggiore chiarezza e professionalità.\\
Abbiamo quindi deciso di proseguire per un'altra soluzione altrettanto valida e, probabilmente, più adeguata 
alla risoluzione del nostro problema. La scelta è stata quella di eseguire una clusterizzazione sulle 
immagini in modo da creare dei \textit{pool} di immagini correlate, cambia quindi l'idea iniziale di 
riconoscere l'animale ma viene mantenuta l'idea di un percorso interattivo ed espositivo.\\
Durante la realizzazione di questa soluzione abbiamo approfondito l'uso di diversi algoritmi di clustering 
e il tuning degli stessi, modificando parametri, numero di clusters, tipologie di dati ecc.
\subsubsection{Sfide informatiche affrontate}
I problemi sono stati molteplici e si sono distribuiti uniformemente durante tutta la realizzazione del 
progetto. Ne abbiamo riscontrati sia durante la prima fase in cui abbiamo cercato di allenare l'AI che 
durante la preprazione del dataset per la clusterizzazione.\\
Inizialemente il problema principale è stato quello di avere delle immagini chiare e simili in modo da 
poter insegnare quali fossero i tratti caratterstici di ciascun animale, ma essi risultavano troppo 
confusionarie e poco equiparabili, alcune erano sfocate, altre tagliate, altre ancora super luminose e 
altre totalmente buie. Un altro problema durante questa fase è stato quello della dimensione delle foto, 
esse risultavano troppo pesanti e dettagliate, quindi ingestibili dalla potenza di calcolo delle macchine 
che abbiamo a disposizione.\\
I problemi sono proseguiti nella fase della clusterizzazione, uno dei lavori più lunghi e macchinosi è 
stato quello di realizzare il dataset. Abbiamo docuto scegliere quale features tenere, quali fossero le 
più interessanti e quelle più facilmente reperibili in modo da avere pochi valori \textit{null}. Una volta 
scelti i valori da studiare abbiamo dovuto capire quale fosse il metodo migliore per ottenerli, abbiamo 
provato diverse tipologie di strumenti e diverse versioni degli stessi (molteplici OCR e diversi lettori 
di filesystem).\\
Infine abbiamo riolto diversi problemi legati alla memorizzazione dei dati in modo corretto, nonostante 
le diverse tecniche testate abbiamo avuto diversi problemi di dati inconsistenti e quindi abbiamo dovuto 
risolvere molti problemi di valori \textit{null} all'interno del dataset. Talvolta abbiamo dovuto anche 
calcolare dei dati in modo da avere più informazioni da analizzare (come le stagioni o le fasi lunari), 
oppure convertirne alcuni in mdo da renderli sensati dal punto di vista della clusterizzazione (con dei 
one-hot-encoder ad esempio).\\
Nonostante queste accortezze non siamo comunque riusciti a creare un dataset con tutti i dati che volevamo 
ma comunque alcuni sono andati persi.
\subsubsection{Rassegna della letteratura}
\textit{DA ESTENDERE}
\subsubsection{Divisione dei compiti nel gruppo}
Il gruppo ha lavorato abbastanza uniformemente nella realizzazione del progetto, tutti hanno partecipato 
in modo differente al completamento dello stesso.\\
Durante la stesura del codice è risultato fondamentale un approccio in cui ognuno fosse libero di 
criticare e consigliare quale fosse la soluzione ideale per lui. Ci sono state molte situazioni in cui 
ci siamo riuniti e abbiamo provato ad ideare e sviluppare delle idee per poter risolvere i problemi 
incontrati nel percorso.\\
\textit{DA ESTENDERE}
\subsubsection{Risultati ottenuti in sintesi}
\textit{DA ESTENDERE}


\newpage
\section{Metodo Proposto}

\subsection{Scelta della Soluzione}
La soluzione scelta è stata quella di creare un percorso interattivo tra immagini correlate in 
grado di coinvolgere e stupire l'utente in modo dinamico. Questa soluzione è stata ottenuta 
eseguendo una clusterizzazione della raccolta di immagini.\\
Una volta raccolte tutte le fotografie, sono state categorizzate e indicizzate in modo da creare 
un dataset di informazioni, da studiare e ordinare per poi proporre un percorso interessante 
all'utente finale.\\
Le soluzioni proposte non sono state poche, abbiamo provato diversi metodi e testato diverse 
tecnologie per ottenere il risultato migliore, quello che coincidesse esattamente con le nostre 
necessità di avere un sistema reattivo e sempre pronto ad essere aggiornato e migliorato con 
altre immagini e informazioni, personalizzabile dall'utente e utile sia come intrattenimento, 
quindi a scopo ludico, che interessante dal punto di vista scientifico e dell'osservazione. Per 
questo abbiamo studiato e ci siamo anche informati su quali potessero essere le correlazioni 
scientifiche con ciò che osservavamo e ciò che valutavamo, in modo da poter avere anche un 
occhio critico per poter valutare se i risultati scovati fossere corretti oppure inutili e 
sbagliati. Non essendo una materia di nostra competenza, infatti, abbiamo dovuto prestare molta 
attenzione.\\
La soluzione corretta è stata scelta anche dopo aver effettuato dell'inferenza sui dati in modo 
da osservare la distribuzione degli stessi e capire se ci fossero stati dei problemi. 
\subsubsection{Alternativi considerati e giustificazioni della scelta}
Inizialmente abbiamo provato a realizzare un'AI in grado di riconoscere l'animale selvatico 
dall'immagine, una sorta di \textit{lens} in grado di analizzare la foto e il contesto per poter 
tracciare quale fosse il soggetto della foto. L'idea è stata abbandonata piuttosto velocemente 
perchè le risorse a disposizione erano molto inferiori a quelle richieste da un lavoro così 
costoso computazionalmente parlando.\\
Per questo abbiamo optato per una soluzione più alla nostra portata e ugualmente molto 
interessante, abbiamo creato un dataset con le immagini a nostra disposizione abbiamo provato 
ad utilizzare diversi metodi di clusterizzazione con diverse features da considerare e diverso 
numero di clusters.\\
Dopo diverse prove e ricerche abbiamo capito che solo empiricamente e provando si può ottenere 
il risultato migliore e più adatto alle esigenze, seguendo anche delle regole teoriche come ad 
esempio l'\textbf{Elbow Method}, un metodo euristico che aiuta a scegliere il numero di 
clusters necessari per ottenere il risultato migliori in termini di scelte di features e numero 
di clusters.\\
\subsubsection{Metodologia per la misurazione delle performance}
La misurazione della performance è avvenuta tramite diversi metodi e in diversi momenti durante la 
realizzazione di questo progetto. Abbiamo sempre cercato di ottenere dei buoni risultati tenendo 
sotto controllo i dati ottenuti.\\
Per questo motivo, durante la creazione del dataset, abbiamo effettuato diversi test e diverse 
misurazioni per vedere se i dati in output fossero sensati e collezionati nel modo corretto senza 
particolari incongruenze.\\
Per il nostro progetto risulta molto complesso parlare di misurazione della performance visto che 
il risultato proposto è soggettivamente buono e non può essere valutato da parametri numerici. 
Generalmente possiamo osservare che i risultati proposti sono di nostro gradimento ma sicuramente 
possono esserci delle personalizzazioni che magari cambiano l'output in modo da ottenerne uno più 
utile al proprio scopo, magari modificando le varibaili della clusterizzazione o il numero di 
clusters.


\newpage
\section{Risultati Sperimentali}

\subsection{Dimostrazione e Tecnologie}
\subsubsection{Istruzione per la dimostrazione}
La dimostrazione che si ottiene non è altro che un percorso tra delle immagini che sono 
evidentemente correlate, secondo le features che abbiamo deciso di inserire nella clusterizzazione, 
ovvero: \textit{[Inserire le features scelte per la clusterizzazione]}.
\subsubsection{Tecnologie e versioni usate (riproducibilità)}
Le immagini sono state scattate e raccolte maualmente da alcune foto-trappole, convertite poi in 
formato \texttt{.jpeg} e caricate su dropbox, tramite il quale è stato possibile lavorare senza 
avere con se la copia fisica delle imamgini stesse.\\ 
Per gli l'algoritmo di clustering è stato utilizzato Python con alcune sue librerie per poter 
gestire e raccogliere i dati.\\
\textit{DA ESTENDERE} 

\subsection{Risultati}
\subsubsection{Risultati della configurazione migliore}
\subsubsection{Studio di ablazione: confronto tra configurazioni}
\subsubsection{Studio di comparazione con letteratura}


\newpage
\section{Discussione e Conclusioni}

\subsection{Discussione dei Risultati}
\subsubsection{Analisi delle performance rispetto alle aspettative}

\subsection{Validità del Metodo}
\subsubsection{Valutazione se il metodo rispetta le aspettative}

\subsection{Limitazione e Maturità}
\subsubsection{Limiti di applicabilità e bias}

\subsection{Lavori Futuri}
\subsubsection{Proposte per avanzare il progetto}

\printbibliography

\end{document}