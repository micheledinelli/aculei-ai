\title{AI-2023}\documentclass[12pt,a4paper,twoside]{article}

\usepackage[utf8]{inputenc}
\usepackage{amsmath}
\usepackage{amsfonts}
\usepackage{amssymb}
\usepackage[left=3cm,right=3cm,top=4cm,bottom=5cm]{geometry}

%------------------------------ font preferences (1) or sc
\usepackage[sc]{mathpazo} 
\usepackage{newpxmath}

%------------------------------ font preferences (2) 
%\usepackage[libertine,cmintegrals,cmbraces,vvarbb]{newtxmath}

\author{Brajucha Filippo, Dinelli Michele, Hanna Youssef}
\title{Report AI}

\begin{document}

\maketitle

\section*{Abstract}
Aculei è un progetto nato per rendere disponibile un archivio fotografico contenente foto di animali scattate 
da fototrappole automatiche collocate nella campagna Umbra. Da un'idea di Tobia Faverio (il fotografo proprietario dell'archivio fotografico)
nasce il progetto Aculei che ha come obiettivo rendere fruibile l'archivio online fornendo un'esperienza interattiva agli utenti guidata
dall'intelligenza artificiale. Aculei vuole infatti rafforzare il legame tra natura e tecnologia che in questo contesto nasce dal momento stesso
nel quale viene scattata una fotografia. Le foto trappole oggetti tencologi e automatici che si legano al contesto naturale circostante immortalando
la natura nel suo stato puro. La volontà è proporre questo concetto fornendo un percorso virtuale tra le fotografie scattate, guidato dall'intelligenza
artificiale, la quale subentra come portavoce di tecnologia, affiancandosi alla natura.       
\newpage

\tableofcontents

\newpage

\section{Dataset}
L'obiettivo di questa sezione è l'estrazione di dati ordinati e corretti, partendo dall'insieme grezzo di immagini 
raccolte a partire dal 2020 per trasformarle in dati manipolabili e significativi.

\subsection{Raccolta dati}
Il primo problema riscontrato è strettamente correlato alla raccolta dei dati

\section{Clustering}

\section{Conclusioni}

\subsection{Raccolta Dati}
Bullet point delle operazioni eseguite per poter eseguire la raccolta dei dati da trattare.
\begin{itemize}
    \item Ottenute le immagini da DropBox
    \item Estrazione dei meta-dati:
    \begin{itemize}
        \item ocr (\textit{easyocr}, \textit{pytesseract})
        \item lettura filesystem (\textit{exiftool}), abbiamo utilizzato il wrapper Python \textbf{PyExifTool}
    \end{itemize}
    \item Estrazione della fase lunare partendo da data e ora utilizzando uno script Python
\end{itemize}


\end{document}