\title{AI-2023}\documentclass[12pt,a4paper,twoside]{article}

\usepackage[utf8]{inputenc}
\usepackage{amsmath}
\usepackage{amsfonts}
\usepackage{amssymb}
\usepackage[left=3cm,right=3cm,top=4cm,bottom=5cm]{geometry}

%------------------------------ font preferences (1) or sc
\usepackage[sc]{mathpazo} 
\usepackage{newpxmath}

%------------------------------ font preferences (2) 
%\usepackage[libertine,cmintegrals,cmbraces,vvarbb]{newtxmath}

\author{Brajucha Filippo, Dinelli Michele, Hanna Youssef}
\title{Report AI}

\begin{document}

\maketitle

\section*{Abstract}
Aculei è un progetto nato per rendere disponibile un archivio fotografico contenente foto di animali scattate 
da fototrappole automatiche collocate nella campagna Umbra. Da un'idea di Tobia Faverio (il fotografo proprietario dell'archivio fotografico)
nasce il progetto Aculei che ha come obiettivo rendere fruibile l'archivio online fornendo un'esperienza interattiva agli utenti guidata
dall'intelligenza artificiale. Aculei vuole infatti rafforzare il legame tra natura e tecnologia che in questo contesto nasce dal momento stesso
nel quale viene scattata una fotografia. Le foto trappole oggetti tencologi e automatici che si legano al contesto naturale circostante immortalando
la natura nel suo stato puro. La volontà è proporre questo concetto fornendo un percorso virtuale tra le fotografie scattate, guidato dall'intelligenza
artificiale, la quale subentra come portavoce di tecnologia, affiancandosi alla natura.       
\newpage

\tableofcontents

\newpage

\section{Dataset}
L'obiettivo di questa sezione è l'estrazione di dati ordinati e corretti, partendo dall'insieme grezzo di immagini 
raccolte a partire dal 2020 per trasformarle in dati manipolabili e significativi.

\subsection{Raccolta dati}
Il primo problema riscontrato è strettamente correlato alla raccolta dei dati
che sono stati forniti in maniera non ordinata e non strutturata. La prima operazione effettuata è stata quella di scaricare le immagini da Dropbox, dopopodiché sono stati presi in considerazioni più approcci per poter estrarre i dati dai file immagine.
\subsubsection{Prima fase}
Il primo approccio adottato è stato quello di utilizzare sistemi di riconoscimento ottico dei caratteri \textit{(OCR)} per estrarre i dati dai file immagine. In particolare è stato sviluppato uno script in Python che utilizza integrando diverse librerie come \textit{easyocr, pytesseract, keras-ocr} per estrarre i dati dai file immagine. Questo tentativo ha permesso di estrarre i dati dai file immagine, ma ha presentato diversi problemi tra cui la scarsa qualità dei dati per alcuni campi e sopratutto i tempi di esecuzione molto lunghi. 
\subsubsection{Seconda fase}
In questa seconda fase è stata mantenuta l'idea di utilizzare gli \textit{OCR}, ma quelli integrati nei dispositivi \textit{MacOS} e \textit{Ios}. Nello specifico è stata implementata una \textit{shortcut} che permette di estrarre i dati dai file immagine e di salvarli in un file \textit{.csv}. Questa soluzione è stata effettivimante più efficace nel riconoscimento dei caratteri, ma presentava tempi di esecuzioni molto lunghi e la necessità di utilizzare dispositivi \textit{MacOS} o \textit{Ios}.

\subsubsection{Terza fase}
La terza fase è stata quella di utilizzare un sistema di lettura dei meta-dati dei file immagine. In particolare è stato utilizzato \textit{exiftool}, un software open source che permette leggere e manipolare meta-dati di immagini immagine. Questa soluzione ha permesso di estrarre i dati in maniera molto più veloce e una qualità dei dati molto più alta. Inoltre è possibile questa soluzione su qualsiasi sistema operativo aumentandone la versatilità.
\subsubsection{Quarta fase}
La terza fase è stata molto precisa, l'unico problema riscontrato è stata la mancanza dei dati relativi alla temperatura nei meta-dati. Questo problema è stato risolto utilizzando un sistema di \textit{OCR, pytesseract} per estrarre i dati, solamente relativi alla temperatura, dalle foto e inserirli nel file \textit{.csv}. In questo modo il \textit{dataset} era quasi completato.
\subsubsection{Quinta fase}
In questa ultima fase il \textit{dataset} presentava delle mancanze per quanto riguarda la temperatura. Per risolvere questo problema è stata effettuata una seconda iterazione con gli \textit{OCR} utilizzando la libreria di \textit{easyocr}. Quest'ultima ha restituito un risultato più accurato e efficace rispetto a \textit{pytesseract} ed è stato possibile terminare la popolazione del \textit{dataset}.
\section{Clustering}

\section{Conclusioni}

\subsection{Raccolta Dati}
Bullet point delle operazioni eseguite per poter eseguire la raccolta dei dati da trattare.
\begin{itemize}
    \item Ottenute le immagini da DropBox
    \item Estrazione dei meta-dati:
    \begin{itemize}
        \item ocr (\textit{easyocr}, \textit{pytesseract})
        \item lettura filesystem (\textit{exiftool}), abbiamo utilizzato il wrapper Python \textbf{PyExifTool}
    \end{itemize}
    \item Estrazione della fase lunare partendo da data e ora utilizzando uno script Python
\end{itemize}


\end{document}