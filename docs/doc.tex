\title{AI-2023}\documentclass[12pt,a4paper,twoside]{article}
\usepackage[utf8]{inputenc}
\usepackage{amsmath}
\usepackage{amsfonts}
\usepackage{amssymb}
\usepackage[left=3cm,right=3cm,top=4cm,bottom=5cm]{geometry}

\author{Brajucha Filippo, Dinelli Michele, Hanna Youssef}
\title{Report AI}

\begin{document}

\maketitle

\section*{Abstract}
Hello

\newpage

\tableofcontents

\newpage

\section{Dataset}
L'obiettivo di questa sezione è l'estrazione di dati ordinati e corretti, partendo dall'insieme grezzo di immagini 
raccolte a partire dal 2020 per trasformarle in dati manipolabili e significativi.

\subsection{Raccolta dati}
Il primo problema riscontrato è strettamente correlato alla raccolta dei dati

\section{Clustering}

\section{Conclusioni}

\subsection{Raccolta Dati}
Bullet point delle operazioni eseguite per poter eseguire la raccolta dei dati da trattare.
\begin{itemize}
    \item Ottenute le immagini da DropBox
    \item Estrazione dei meta-dati:
    \begin{itemize}
        \item ocr (\textit{easyocr}, \textit{pytesseract})
        \item lettura filesystem (\textit{exiftool}), abbiamo utilizzato il wrapper Python \textbf{PyExifTool}
    \end{itemize}
    \item Estrazione della fase lunare partendo da data e ora utilizzando uno script Python
\end{itemize}


\end{document}